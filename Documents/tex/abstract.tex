\begin{abstract}
  With the increasing amount of sensitive data that is relayed on the internet, the requirement for secure hashing algorithms has grown exponentially.
  Hashing functions are used for many cryptographic purposes, namely a signatory's authenticity, data integrity, key derivation, and pseudorandom bit generation.
  
  In this report we first present the  Merkle-Damg\r{a}rd construction, a legacy pattern used to develop hashing algorithms, such as MD5, SHA1, and many others.
  
  After which will be explained what causes a cryptographic hash function to be considered broken. Various weaknesses of constructions based on the Merkle-Damg\r{a}rd construction will be exposed, with particular attention given to their vulnerability to differential crytptanalysis.
  
  Finaly we will discuss the need for a new family of hashing algorithms, based on a different model: the sponge construction. And we will define the SHA-3 family of hashing algorithms, based on this construction and an underlying permutation \textsc{Keccak}, which have been approved by the NIST.
  \\
  
  \textbf{Keywords}: One-way Hashing Functions, Merkle-Damg\r{a}rd Construction, MD5, SHA1, Brute Force Attack, Birthday Paradox, Differential Cyptanalysis, Merkle-Damg\r{a}rd Weakness, Sponge Function, Keccak, SHA3.

\end{abstract}
