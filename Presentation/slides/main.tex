%
% Permission is granted to copy, distribute and/or modify this
% document under the terms of the Creative Common by-nc-sa License
% version 3.0 (CC BY-NC-SA 3.0). A copy of the license can be found at
% http://creativecommons.org/licenses/by-nc-sa/3.0/legalcode.
%

\usepackage[french]{babel}
\usepackage{tikz}
\usetikzlibrary{shapes}
\usetikzlibrary{positioning}
\usepackage{color}

% Highlight macros
\newcommand{\highlight}[1]{\textcolor{structure.fg}{\bfseries #1}}

%% Title, subtitle, authors, institute, date, ...
\title{Implémentation de fonctions en éponge}

\author[Amélie Guémon\\Ida Tucker]{Amélie Guémon\\Ida Tucker\\[-.25em]
\texttt{\scriptsize <amelie.guemon@etu.u-bordeaux.fr>}\\}

\institute[Master CSI, France]{Master CSI, Université de Bordeaux, France}

\date{\today}

%%%%%%%%%%%%%%%%%%%%%%%%%%[ Document ]%%%%%%%%%%%%%%%%%%%%%%%%%%
\begin{document}

\begin{frame}
  \vspace{3.5em}
  \titlepage

  \begin{center}
    \includegraphics[scale=.2]{cc-by-nc-sa.pdf}
  \end{center}
\end{frame}

\begin{frame}
  \frametitle{Plan}
  \tableofcontents[subsectionstyle=hide]
\end{frame}

%%%%%%%%%%%%%%%%%%%%%%%%%%%%[ Introduction ]%%%%%%%%%%%%%%%%%%%%%%%%%%


%TODO slide definition avec schema fonction de hashage
\begin{frame}[fragile]
  \frametitle{Définition: fonction de hashage}

Une fonction de hashage est une application qui associe à un ensemble de départ infini $\{0,1\}^*$ un ensemble d'arrivée fini  $\{0,1\}^n$ constitué de chaînes de bits de taille n.

\end{frame}

\begin{frame}[fragile]
  \frametitle{Propriétés requises}
  \vfill
  \begin{itemize}
  \item \textbf{Résistance à la Pré-image}: Pour un hash $y$ donné, il est dur de trouver une pré-image $x \in f^{-1}(H)$ tel que $y = H(x)$.
  \item \textbf{Résistance à la Seconde Pré-image}: Pour un clair $x$, il est dur de trouver un autre clair $x',\ x'\neq x$ tel que $H(x) = H(x')$.
  \item \textbf{Résistance aux Collisions}: Il est dur de trouver 2 messages clairs $x$ et $x'$ avec $x \neq x'$ tel que $H(x) = H(x')$.
  \end{itemize}
  \vspace{0.8cm}
  Résistance aux collisions $\Rightarrow$ Résistance à la seconde pré-image $\Rightarrow$ Résistance à la première pré-image
  \vfill
\end{frame}

\begin{frame}[fragile]
  \frametitle{Padding}
  \vfill
  \begin{itemize}
  \item \textbf{Padding Simple}: Représenté par $10*$, il faut rajouter un $1$, puis un nombre fini de $0$, de telle sorte que la longueur du resultat soit un multiple de le taille des blocks que l'on doit utiliser.
  \end{itemize}
  \vfill
  \begin{figure}[H]
        \centering
        \begin{tikzpicture}[scale=1]
    
         \draw [name=green, fill=red!70!grey, line width=2pt] (0,0) rectangle (4,0.4);
         \draw [fill=green!80, line width=2pt] (4,0) rectangle (4.3,0.4);
         \draw [fill=green!80, line width=2pt] (4.3,0) rectangle (6.3,0.4);

         \node [align=center] at (2,0.2){\textbf{M}};
         \node [align=center] at (4.15,0.2){\textbf{1}};
         \node [align=left]   at (4.8,0.2){\textbf{00}$\ldots$};
    
        \draw [<->, >=latex, line width=1pt, color=red!70!grey] (0,-0.2) -- (4,-0.2);
        \draw [<->, >=latex, line width=1pt, color=grey] (0,-1) -- (6.3,-1);

        \node [align=center, color=red!70!grey] at (2,-0.5){$\vert$ \textbf{M} $\vert$};
        \node [align=center, color=grey] at (3,-1.3){\textbf{Multiple of r\ bits}};
    
  \end{tikzpicture}
  \caption{\label{fig:SimplePadding}Simple padding.}
\end{figure}
\vfill
\end{frame}

\begin{frame}[fragile]
  \frametitle{Padding}
  \vfill
  \begin{itemize}
  \item \textbf{Merkle-Damg\r{a}rd Padding}: Représenté par $10*1|M|$, il faut rajouter un $1$, puis un nombre fini de $0$, de telle sorte que la longueur du resultat soit congru à $448$ mod $512$. Ensuite, on y ajoute la longueur du message, sur $64$ bits.
  \end{itemize}
  \vfill
  \begin{figure}[H]
        \centering
        \begin{tikzpicture}[scale=1.2]
    
        \draw [name=green, fill=red!70!grey, line width=2pt] (0,0) rectangle (4,0.4);
        \draw [fill=green!80, line width=2pt] (4,0) rectangle (4.3,0.4);
        \draw [fill=green!80, line width=2pt] (4.3,0) rectangle (6.3,0.4);
        \draw [fill=cyan!50!blue, line width=2pt] (6.3,0) rectangle (7.9,0.4);

        \node [align=center] at (2,0.2){\textbf{M$_{k-1}$}};
        \node [align=center] at (4.15,0.2){\textbf{1}};
        \node [align=left]   at (4.8,0.2){\textbf{00}$\ldots$};
        \node [align=center] at (7.1,0.15){$\vert \textbf{M}\vert $};
    
        \draw [<->, >=latex, line width=1pt, color=red!70!grey] (0,-0.2) -- (4,-0.2);
        \draw [<->, >=latex, line width=1pt, color=green!80] (0,0.7) -- (6.3,0.7);
        \draw [<->, >=latex, line width=1pt, color=grey] (0,-1) -- (7.9,-1);
        \draw [<->, >=latex, line width=1pt, color=cyan!50!blue] (6.3,-0.2) -- (7.9,-0.2);

        \node [align=center, color=red!70!grey] at (2,-0.5){$\vert$ \textbf{M$_{k-1}$} $\vert$};
        \node [align=center, color=green!80] at (3.15,1){\textbf{448\ mod\ 512}};
        \node [align=center, color=grey] at (4,-1.3){\textbf{512\ bits}};
        \node [align=center, color=cyan!50!blue] at (7.1,-0.5){\textbf{64\ bits}};
    
        \end{tikzpicture}
    \caption{\label{fig:MDPadding}Merkle-Damg\r{a}rd padding.}
    \end{figure}
\vfill
\end{frame}

%%%%%%%%%%%%%%%%%%%%%%%%%%%%%%%%%%%%%%%%%%%%%%%%%%%%%%%%%%%%%%%%%%%%%%
\section{Merkle-Damg\r{a}rd et ses applications}

\begin{frame}<handout:0>
  \frametitle{Plan}
  \tableofcontents[currentsection,subsectionstyle=hide]
\end{frame}

\begin{frame}[fragile]
  \frametitle{Construction de Merkle-Damg\r{a}rd}
  
  La construction de Merkle-Damg\r{a}rd permet de définir des fonctions de hachage en itérant des fonctions de compression.
  \begin{itemize}
  \item{Une fonction de compression part d'un ensemble fini vers un ensemble fini.}
  \item{Une fonction de hachage part d'un ensemble infini vers un ensemble fini.}
  \end{itemize}
\end{frame}

\begin{frame}[fragile]
  \frametitle{Construction de Merkle-Damg\r{a}rd}
  % SCHEMA %
  \begin{itemize}
\item{Théorème: Si la fonction de compression $h$ utilisée par la fonction de hachage $H$ l'est aussi.} 
  \end{itemize}
\end{frame}

\begin{frame}[fragile]
  \frametitle{Applications}
  % SCHEMA ? %
  \begin{itemize}
  \item{SHA1}
  \item{MD5} 
  \end{itemize}
\end{frame}
%%%%%%%%%%%%%%%%%%%%%%%%%%%%%%%%%%%%%%%%%%%%%%%%%%%%%%%%%%%%%%%%%%%%%%
\section{Faiblesse de Merkle-Damg\r{a}rd}

\begin{frame}<handout:0>
  \frametitle{Plan}
  \tableofcontents[currentsection]
\end{frame}

\begin{frame}
  \frametitle{On continue !!!}

  \vfill
  Ce slide est quasiment vide !
  \vfill

\end{frame}


%%%%%%%%%%%%%%%%%%%%%%%%%%%%%%%%%%%%%%%%%%%%%%%%%%%%%%%%%%%%%%%%%%%%%%
\section{Fonctions de hachage en éponge}

\begin{frame}<handout:0>
  \frametitle{Plan}
  \tableofcontents[currentsection,subsectionstyle=hide]
\end{frame}

\nocite{*}
\bibliographystyle{alpha}

\begin{frame}[allowframebreaks]
  \frametitle{Livres et références}
  \bibliography{bibliography}
\end{frame}

%%%%%%%%%%%%%%%%%%%%%%%%%%%%%%%%%%%%%%%%%%%%%%%%%%%%%%%%%%%%%%%%%%%%%%
\begin{frame}
  \vfill
  \centering
  \highlight{\Huge Questions~?}
  \vfill
\end{frame}
