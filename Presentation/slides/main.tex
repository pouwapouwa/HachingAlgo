%
% Permission is granted to copy, distribute and/or modify this
% document under the terms of the Creative Common by-nc-sa License
% version 3.0 (CC BY-NC-SA 3.0). A copy of the license can be found at
% http://creativecommons.org/licenses/by-nc-sa/3.0/legalcode.
%

\usepackage[french]{babel}
\usepackage{tikz}
\usetikzlibrary{shapes}
\usetikzlibrary{positioning}
\usepackage{color}
\usepackage{amsmath, amsthm, amscd, amssymb, amsfonts, amsxtra}
\usepackage{stmaryrd}
\usepackage{graphicx, wrapfig, lipsum}

\tikzset{
  every overlay node/.style={
    %draw=black,fill=white,rounded corners,
    anchor=north west, inner sep=0pt,
  },
}
% Usage:
% \tikzoverlay at (-1cm,-5cm) {content};
% or
% \tikzoverlay[text width=5cm] at (-1cm,-5cm) {content};
\def\tikzoverlay{
   \tikz[remember picture, overlay]\node[every overlay node]
}

% Highlight macros
\newcommand{\highlight}[1]{\textcolor{structure.fg}{\bfseries #1}}

%% Title, subtitle, authors, institute, date, ...
\title{Implémentation de fonctions en éponge}

\author[Amélie Guémon \& Ida Tucker]{Amélie Guémon\\Ida Tucker\\[.25em]
\texttt{\scriptsize <amelie.guemon@etu.u-bordeaux.fr>}\\[-.25em]
\texttt{\scriptsize <ida.tucker@etu.u-bordeaux.fr>}\\}

\institute[Master CSI]{Master CSI, Université de Bordeaux, France}

\date{\today}

%%%%%%%%%%%%%%%%%%%%%%%%%%[ Document ]%%%%%%%%%%%%%%%%%%%%%%%%%%
\begin{document}

\begin{frame}
  \vspace{3.5em}
  \titlepage

  \begin{center}
    \includegraphics[scale=.2]{cc-by-nc-sa.pdf}
  \end{center}
\end{frame}

\begin{frame}
  \frametitle{Plan}
  \tableofcontents[subsectionstyle=hide]
\end{frame}

%%%%%%%%%%%%%%%%%%%%%%%%%%%%[ Introduction ]%%%%%%%%%%%%%%%%%%%%%%%%%%

\begin{frame}[fragile]
  \frametitle{Définition: fonction de hachage}
  \vfill

\centerline{H:$\{0,1\}^* \rightarrow \{0,1\}^n$}

    \vfill

    \begin{figure}[H]
        \centering
        \begin{tikzpicture}[scale=1,ele/.style={fill=black,circle,minimum width=.8pt,inner sep=1pt},every fit/.style={ellipse,draw,inner sep=-2pt}]

        \draw (0,2) ellipse (1cm and 2cm);
        \draw (4,2) ellipse (1cm and 1.4cm);

        \node[ele,label=left:$a$] (a1) at (0.2,3.5) {};    
        \node[ele,label=left:$b$] (a2) at (0.2,2.5) {};    
        \node[ele,label=left:$c$] (a3) at (0.2,1.5) {};
        \node[ele,label=left:$d$] (a4) at (0.2,0.5) {};

        \node[ele,label=right:$1$] (b1) at (4,2.9) {};
        \node[ele,label=right:$2$] (b2) at (4,2) {};
        \node[ele,label=right:$3$] (b3) at (4,1.1) {};

        \draw[->,thick,shorten <=2pt,shorten >=2pt] (a1) -- (b3);
        \draw[->,thick,shorten <=2pt,shorten >=2] (a2) -- (b2);
        \draw[->,thick,shorten <=2pt,shorten >=2] (a3) -- (b1);
        \draw[->,thick,shorten <=2pt,shorten >=2] (a4) -- (b2);
          
        \end{tikzpicture}
        \caption{Surjectivité d'une fonction de Hachage}
    \end{figure}

    \vfill

\end{frame}

\begin{frame}[fragile]
  \frametitle{Propriétés requises}  
  \vfill
          \begin{itemize}
          \item \textbf{Résistance à la Pré-image}: Pour une empreinte $y$ donné, il est dur de trouver une pré-image $x \in f^{-1}(H)$ tel que $y = H(x)$.
          \item \textbf{Résistance à la Seconde Pré-image}: Pour un clair $x$, il est dur de trouver un autre clair $x',\ x'\neq x$ tel que $H(x) = H(x')$.
          \item \textbf{Résistance aux Collisions}: Il est dur de trouver 2 messages clairs $x$ et $x'$ avec $x \neq x'$ tel que $H(x) = H(x')$.
          \end{itemize}
\vfill
  \begin{figure}[H]
        \centering
        \begin{tikzpicture}[scale=1]
          \draw  [rounded corners] (-6,0) rectangle (-3,2);
          \node [align=center] at (-4.5,1.5){\textbf{Résistance}};
          \node [align=center] at (-4.5, 1){\textbf{aux}};
          \node [align=center] at (-4.5,0.5){\textbf{Collisions}};
          
          \draw  [rounded corners] (-1.5,0) rectangle (1.5,2);
          \node [align=center] at (0,1.5){\textbf{Résistance}};
          \node [align=center] at (0, 1){\textbf{à la seconde}};
          \node [align=center] at (0,0.5){\textbf{Pré-image}};
          
          \draw  [rounded corners] (3,0) rectangle (6,2);
          \node [align=center] at (4.5,1.5){\textbf{Résistance}};
          \node [align=center] at (4.5, 1){\textbf{à la première}};
          \node [align=center] at (4.5,0.5){\textbf{Pré-image}};
          
        \draw [->, >=latex, double, line width=1pt] (-3,1) -- (-1.5,1);
        \draw [->, >=latex, double, line width=1pt] (1.5,1) -- (3,1);
    \end{tikzpicture}
\end{figure}
\vfill
\end{frame}

\begin{frame}[fragile]
  \frametitle{Padding}
  \vfill
  \begin{itemize}
  \item \textbf{Merkle-Damg\r{a}rd Padding}: Représenté par $10*|M|$, il faut rajouter un $1$, puis un nombre fini de $0$, de telle sorte que la longueur du resultat soit congru à $448$ mod $512$. Ensuite, on y ajoute la longueur du message, sur $64$ bits.
  \end{itemize}
  \vfill
  \begin{figure}[H]
        \centering
        \begin{tikzpicture}[scale=1.2]
    
        \draw [name=green, fill=red!70!grey, line width=2pt] (0,0) rectangle (4,0.4);
        \draw [fill=green!80, line width=2pt] (4,0) rectangle (4.3,0.4);
        \draw [fill=green!80, line width=2pt] (4.3,0) rectangle (6.3,0.4);
        \draw [fill=cyan!50!blue, line width=2pt] (6.3,0) rectangle (7.9,0.4);

        \node [align=center] at (2,0.2){\textbf{M$_{k-1}$}};
        \node [align=center] at (4.15,0.2){\textbf{1}};
        \node [align=left]   at (4.8,0.2){\textbf{00}$\ldots$};
        \node [align=center] at (7.1,0.15){$\vert \textbf{M}\vert $};
    
        \draw [<->, >=latex, line width=1pt, color=red!70!grey] (0,-0.2) -- (4,-0.2);
        \draw [<->, >=latex, line width=1pt, color=green!80] (0,0.7) -- (6.3,0.7);
        \draw [<->, >=latex, line width=1pt, color=grey] (0,-1) -- (7.9,-1);
        \draw [<->, >=latex, line width=1pt, color=cyan!50!blue] (6.3,-0.2) -- (7.9,-0.2);

        \node [align=center, color=red!70!grey] at (2,-0.5){$\vert$ \textbf{M$_{k-1}$} $\vert$};
        \node [align=center, color=green!80] at (3.15,1){\textbf{448\ mod\ 512}};
        \node [align=center, color=grey] at (4,-1.3){\textbf{512\ bits}};
        \node [align=center, color=cyan!50!blue] at (7.1,-0.5){\textbf{64\ bits}};
    
        \end{tikzpicture}
    \caption{Merkle-Damg\r{a}rd padding.}
    \end{figure}
\vfill
\end{frame}

%%%%%%%%%%%%%%%%%%%%%%%%%%%%%%%%%%%%%%%%%%%%%%%%%%%%%%%%%%%%%%%%%%%%%%
\section{Merkle-Damg\r{a}rd et ses applications}

\begin{frame}<handout:0>
  \frametitle{Plan}
  \tableofcontents[currentsection,subsectionstyle=hide]
\end{frame}

\begin{frame}[fragile]
  \frametitle{Construction de Merkle-Damg\r{a}rd}
  \vfill
 Merkle-Damg\r{a}rd définit des fonctions de hachage en \textbf{itérant des fonctions de compression}.
   \vfill
  \begin{itemize}
  \item{\textbf{Fonction de compression:} part d'un ensemble fini vers un ensemble fini réduit.}
  \vfill
  \item{\textbf{Fonction de hachage:} part d'un ensemble infini vers un ensemble fini.}
  \end{itemize}
  \vfill
\end{frame}

\begin{frame}[fragile]
  \frametitle{Construction de Merkle-Damg\r{a}rd}
  \begin{figure}[ht]
        \centering
        \begin{tikzpicture}[scale=0.7]

                \node at (-0.7,3.7){$M_{0}$};
                \draw[->, >=latex, line width=1pt] (-0.5,3.3) -- (-0.5,1.5) -- (0,1.5);
                \node[align=center] at (-1.5,0.5){IV\\(size n)};
                \draw[->, >=latex, line width=1pt] (-1,0.7) -- (0,0.7);
                \draw[line width=1pt] (0,0) -- (0,2.8) -- (1.5,1.5) -- (1.5,0) -- cycle;
                \node at(0.7,1){$h$};

                \node at (1.7,3.7){$M_{1}$};
                \draw[->, >=latex, line width=1pt] (2,3.3) -- (2,1.5) -- (2.5,1.5);
                \draw[->, >=latex, line width=1pt] (1.5,0.7) -- (2.5,0.7);
                \draw[line width=1pt] (2.5,0) -- (2.5,2.8) -- (4,1.5) -- (4,0) -- cycle;
                \node at(3.2,1){$h$};
                \draw[->, >=latex, line width=1pt] (4,0.7) -- (5,0.7);

                \draw[dashed] (5.1,0.7) -- (5.8,0.7);
                \draw[dashed] (2.5,3.5) -- (5.5,3.5);

                \node at (6.2,3.7){$M_{k-1}$};
                \draw[->, >=latex, line width=1pt] (6.5,3.3) -- (6.5,1.5) -- (7,1.5);
                \draw[->, >=latex, line width=1pt] (6,0.7) -- (7,0.7);
                \draw[line width=1pt] (7,0) -- (7,2.8) -- (8.5,1.5) -- (8.5,0) -- cycle;
                \node at(7.7,1){$h$};

                \node at (8.7,3.7){$|M|$};
                \draw[->, >=latex, line width=1pt] (9,3.3) -- (9,1.5) -- (9.5,1.5);
                \draw[->, >=latex, line width=1pt] (8.5,0.7) -- (9.5,0.7);
                \draw[line width=1pt] (9.5,0) -- (9.5,2.8) -- (11,1.5) -- (11,0) -- cycle;
                \node at(10.2,1){$h$};
                \draw[->, >=latex, line width=1pt] (11,0.7) -- (12,0.7);
                \node[align=center] at (12.7,0.5){H(M)\\(size n)};

        \end{tikzpicture}
        \caption{\label{fig:constructionMD}Merkle-Damg\r{a}rd construction.}
  \end{figure}

  \begin{itemize}
        \item{\textbf{Théorème:} Si la fonction de compression $h$ utilisée par la fonction de hachage $H$ est résistante aux collisions, alors $H$ l'est aussi.} 
  \end{itemize}
  \vfill
\end{frame}

\begin{frame}[fragile]
  \frametitle{Applications}
  \begin{itemize}
  \item{MD5:$\{0,1\}^* \rightarrow \{0,1\}^{128}$}
  \item{SHA1:$\{0,1\}^* \rightarrow \{0,1\}^{160}$} 
  \end{itemize}

  \begin{figure}[!ht]
        \centering
        \begin{tikzpicture}[scale=0.8]
        %Box Messages
        \draw  (0,0) rectangle (1,0.5); \node [align=center] at (0.5,0.25){\textbf{A}};
        \draw  (1,0) rectangle (2,0.5); \node [align=center] at (1.5,0.25){\textbf{B}};
        \draw  (2,0) rectangle (3,0.5); \node [align=center] at (2.5,0.25){\textbf{C}};
        \draw  (3,0) rectangle (4,0.5); \node [align=center] at (3.5,0.25){\textbf{D}};
        \draw  (4,0) rectangle (5,0.5); \node [align=center] at (4.5,0.25){\textbf{E}};
        \draw  (0,6) rectangle (1,6.5); \node [align=center] at (0.5,6.25){\textbf{A}};
        \draw  (1,6) rectangle (2,6.5); \node [align=center] at (1.5,6.25){\textbf{B}};
        \draw  (2,6) rectangle (3,6.5); \node [align=center] at (2.5,6.25){\textbf{C}};
        \draw  (3,6) rectangle (4,6.5); \node [align=center] at (3.5,6.25){\textbf{D}};
        \draw  (4,6) rectangle (5,6.5); \node [align=center] at (4.5,6.25){\textbf{E}};
        %Cercles
        \draw  (4.5,5) circle (0.25); \node [align=center] at (4.5,5){+};
        \draw  (4.5,4) circle (0.25); \node [align=center] at (4.5,4){+};
        \draw  (4.5,3) circle (0.25); \node [align=center] at (4.5,3){+};
        \draw  (4.5,2) circle (0.25); \node [align=center] at (4.5,2){+};
        %Boxs + Textes
        \draw  [rounded corners] (2.7,4.7) rectangle (3.3,5.3); \node [align=center] at (3,5){F};
        \draw  [rounded corners] (0.35,3.8) rectangle (1.25,4.2); \node [align=center] at (0.75,4){$\ll_5$};
        \draw  [rounded corners] (1,2.8) rectangle (2,3.2); \node [align=center] at (1.5,3){$\ll_{30}$};
        %Traits et Fleches
        \draw [->, >=latex] (0.25,6) -- (0.25,1.5) -- (1.5,0.75) -- (1.5,0.5);
        \draw [-] (4.5,6) -- (4.5,5.25);\draw [-] (4.5,4.75) -- (4.5,4.25);\draw [-] (4.5,3.75) -- (4.5,3.25);\draw [-] (4.5,2.75) -- (4.5,2.25);\draw [->, >=latex] (4.5,1.75) -- (4.5,1.5) -- (0.5,0.75) -- (0.5,0.5);
        \draw [-] (1.25,6) -- (1.25,5) -- (1.5,5) -- (1.5,4.1);\draw [-] (1.5,3.905) -- (1.5,3.2);\draw [->, >=latex] (1.5,2.8) -- (1.5,1.5) -- (2.5,0.75) -- (2.5,0.5);
        \draw [-] (2.25,6) -- (2.25,5.59);\draw [-] (2.25,5.4) -- (2.25,4.1);\draw [->, >=latex] (2.25,3.905) -- (2.25,1.5) -- (3.5,0.75) -- (3.5,0.5);
        \draw [-] (3.75,6) -- (3.75,5.085);\draw [-] (3.75,4.88) -- (3.75,4.1);\draw [->, >=latex] (3.75,3.905) -- (3.75,1.5) -- (4.5,0.75) -- (4.5,0.5);
        \draw [-] (0.75,6) -- (0.75,4.2);\draw [-] (1.25,4) -- (4.25,4);
        \draw [-] (1.75,6) -- (1.75,5.5) -- (2.9,5.5) -- (2.9,5.3);\draw [-] (3.3,6) -- (3.3,5.5) -- (3.1,5.5) -- (3.1,5.3);\draw [-] (3.3,5) -- (4.25,5);
        %Arcs de cercle
        \draw (2.25,5.59) arc(45:-45:0.15cm);\draw (2.25,4.1) arc(45:-45:0.15cm);\draw (1.5,4.1) arc(45:-45:0.15cm);\draw (3.75,4.1) arc(45:-45:0.15cm);\draw (3.75,5.085) arc(45:-45:0.15cm);

        \end{tikzpicture}
        \caption{\label{fig:SHA-1}Le $i^{ème}$ tour de la fonction de compression de SHA-1 $(0\le i \le 79)$.}
\end{figure}

\end{frame}
%%%%%%%%%%%%%%%%%%%%%%%%%%%%%%%%%%%%%%%%%%%%%%%%%%%%%%%%%%%%%%%%%%%%%%
\section{Attaques contre Merkle-Damg\r{a}rd}

\begin{frame}<handout:0>
  \frametitle{Plan}
  \tableofcontents[currentsection]
\end{frame}

\begin{frame}
  \frametitle{Attaque par force brute}
  \vfill
\textbf{Objectif}: Trouver des collisions pour une fonction de hachage $$H: \{0,1\}^* \rightarrow \{0,1\}^n$$

\textbf{Algorithme}:
\begin{itemize}
        \item Choisir de façon aléatoire une ensemble $\mathcal{E}$ de messages dans $\{0,1\}^*$ de cardinal $K$.
        \item Tester pour tous les couples de messages ($M$, $M'$) $\in \mathcal{E}^2$ tels que $M \neq M'$ si $H(M)=H(M')$.
\end{itemize}
  \vfill
\end{frame}

\begin{frame}
  \frametitle{Probabilité de succès}
  \vfill
  \textbf{Si}:
  \begin{itemize}
        \item $\mathcal{E}$ est choisi de façon aléatoire et uniforme parmi tous les messages possibles.
        \item $N = \# \{0,1\}^n = 2^n $ est le nombre total de condensés possibles.
  \end{itemize}
  \vspace{1cm}
  \textbf{Alors}:\\ 
  La probabilité de trouver des collisions au bout de $K=\# \mathcal{E}$ essais ne dépend que de la taille du condensé.\\
  \vspace{0.5cm} 
Pour une probabilité de trouver des collisions \textbf{$>50\%$}:
\begin{equation}
   \fbox{$
   \begin{array}{rcl}
      K & \approx 1.18 \times \sqrt{N}\\
   \end{array}
   $}
   \end{equation}
  \vfill
\end{frame}

\begin{frame}
  \frametitle{Application numérique}
  \vfill

\textbf{Application à MD5 et SHA1}:
\begin{itemize}
\item taille du condensé de MD5: 128 bits.
$$P_{Success} > 50\% \mbox{ pour } 2^{64} \mbox{ calculs de condensés.}$$
\item taille du condensé de SHA1: 160 bits.
$$P_{Success} > 50\% \mbox{ pour } 2^{80} \mbox{ calculs de condensés.}$$
\end{itemize}
  \vspace{0.3cm}
\textbf{Recommandations}: $n \ge 128$, voir $n\ge 160$
  \vfill
\end{frame}

\begin{frame}[fragile]
  \frametitle{Fonctions de hachage cassées}
  \vfill
\textbf{Définition}:\\
Une fonction de hachage est dite \emph{cassée} lorsqu'il existe une attaque connue permettant de trouver des collisions ayant une complexité moindre que l'attaque par force brute.
  \vspace{1.5cm}
  
\textbf{État actuel}:\\
\begin{itemize}
\item{La \emph{cryptanalyse différentielle} a cassé de nombreuses fonctions de hachage itérées, basées sur Merkle-Damg\r{a}rd (dont SHA0, SHA1, MD5).}
\item{Des attaques génériques contre Merkle-Damg\r{a}rd existent. }
\end{itemize}
  \vfill
\end{frame}


%%%%%%%%%%%%%%%%%%%%%%%%%%%%%%%%%%%%%%%%%%%%%%%%%%%%%%%%%%%%%%%%%%%%%%
\section{Fonctions de hachage en éponge}

\begin{frame}<handout:0>
  \frametitle{Plan}
  \tableofcontents[currentsection,subsectionstyle=hide]
\end{frame}

\begin{frame}
  \frametitle{Etat d'une fonction en éponge}
  \vfill
  \begin{itemize}
  \item $S$: l'état (state) de la fonction $S = R \vert \vert C$.
    \item $R$: partie externe de l'état, les premier $r$ bits de l'état.
  \item $C$: partie interne de l'état, de taille $c=b-r$ bits.
  \end{itemize}
 \begin{figure}[H]
        \centering
        \begin{tikzpicture}[scale=0.7]
    
        \draw [fill=red!70!grey, line width=2pt] (-0.5,-3) rectangle (0.5,-1);
        \draw [fill=green!80, line width=2pt] (-0.5,-1) rectangle (0.5,3);
        
        \node [align=center] at (0,1){\textbf{$R$}};
        \node [align=center]   at (0,-2){\textbf{$C$}};

        \node [align=center] at (-1.3,1){\textbf{$r$}};        
        \node [align=center] at (-2.2,-2){\textbf{$c =b-r$}};        

        \node [align=center] at (1.4,0){\textbf{$b$}}; 
        
        \draw [<->, >=latex, line width=1pt, color=red!70!grey] (-1,-3) -- (-1,-1);
        \draw [<->, >=latex, line width=1pt, color=green!80] (-1,-1) -- (-1,3);
        \draw [<->, >=latex, line width=1pt, color=grey] (1,-3) -- (1,3);
    
        \end{tikzpicture}
    \caption{État d'une fonction en éponge.}
    \end{figure}
  \vfill
\end{frame}

\begin{frame}
  \frametitle{Construction en éponge}
  \vfill
\begin{figure}[H]
\centering
\includegraphics[scale=0.4]{sponge1.png}
\caption{Construction de fonctions en éponge $Z = \textsc{sponge}\lbrack f , pad, r\rbrack (M, l)$}
\end{figure}
  \vfill
\end{frame}

\begin{frame}
  \frametitle{Keccak formellement}
  \vfill
  
    \textbf{Paramètres de la fonction \textsc{Keccak}$\lbrack b, n_r \rbrack$ }:\\
  \begin{itemize}
  \item{ \textbf{$w$}: la longueur fixée des chaines de bits permutés.
  
  \hspace{1cm}$w=2^l$ bits, avec $0 \le l \le 6$.
  }
  \item{ \textbf{$n_r$}: le nombre d'itérations effectuées des routines internes.
  
  \hspace{1cm}$n_r = 12+2\times l$.
  }
  \end{itemize}
  \vspace{1cm}
  \textbf{État de la fonction \textsc{Keccak}}
\begin{itemize}
\item{État constitué de $b = 5 \times 5 \times w$ bits.}
\item{État représenté sous forme matricielle:
\begin{itemize}
\item $5\times w$ lignes indexées horizontalement par x, $0 \le x < 5$.
\item $5 \times w$ colonnes indexées verticalement par y, $0 \le y < 5$.
\item $25$ mots (ou rangées) indexées en profondeur par z, $0 \le z < w$.
\end{itemize}
}
\item{$A\lbrack x,y,z\rbrack$ permet d'accéder à tous les bits de l'état.}
\end{itemize}

  \vfill
\end{frame}

\begin{frame}
  \frametitle{État de la fonction Keccak:\\ Représentation matricielle}
  \vfill
\begin{figure}[H]
\centering
\includegraphics[scale=0.3]{StateArray.png}
\caption{État manipulé sous forme matricielle A avec $x,y,z \in\llbracket 0,5 \llbracket \times \llbracket 0,5 \llbracket \times \llbracket 0,w \llbracket$}
\end{figure}
  \vfill
\end{frame}

\begin{frame}
  \frametitle{Une Ronde de \textsc{Keccak}: \textsc{Rnd}}
  \vfill
  \textsc{Keccak}: Pour i de 1 à 24 faire: 
$$\textsc{Rnd}(A,i_r)=\iota ( \chi ( \pi ( \rho (\Theta (A)))), i_r)$$
\vspace{1cm}
Les routines d'une ronde \textbf{Rnd}:
  \begin{itemize}
  \item{La routine $\Theta$}
  \item{La routine $\rho$}
  \item{La routine $\pi$}
  \item{La routine $\chi$}
  \item{La routine $\iota$}
   \end{itemize}
  \vfill
\end{frame}

\begin{frame}
  \frametitle{SHA-3: Généralités}
  \vfill
\begin{itemize}
\item{\textbf{2 novembre 2007:} Publication par le National Institute of         Standards and Technology (NIST) annonçant la recherche d'algorithmes candidats pour une nouvelle famille de fonctions de hachage cryptographiques: SHA-3}
  \vfill
        \item{
        \textbf{Exigences générales}:
                \begin{itemize}
                \item{Condensés de 224, 256, 384, 512 bits }
                \item{Fournir une alternative à SHA-2 (bien que SHA-2 soit encore                 utilisable).}
                \item{Processus similaire à la compétition AES}
                \end{itemize}
        }
\end{itemize}
\vfill
\end{frame}

\begin{frame}
\vfill
   
\centerline{\textbf{SHA-3 implémente \textsc{Keccak}$\lbrack 1600, 24 \rbrack$}}
 
\bgroup
\def\arraystretch{1.5}
  \begin{table}
\begin{tabular}{l | c | c | c }
$w$ & $l$ & $n_r$ & $b$ \\
\hline
$2^6$ & $6$ & $24$ & $1600 $ 
\end{tabular}
\caption{Paramètres de la fonction \textsc{Keccak} pour SHA-3}

\end{table}

\egroup

\vfill

\textbf{Taille des condensés:}
\begin{itemize}
\item{Fonctions de hachage: 224, 256, 384 et 512 bits}
\item{Fonctions à sortie de taille variable: \textsc{SHAKE128}$(M,l)$ et \textsc{SHAKE128}$(M,l)$  pour une capacity de 256 bits, et une sortie de longueur $l$) }
\end{itemize}
\vspace{1cm}
\vfill
\end{frame}


\chapter*{Conclusion}

Our research presents the evolution of hash functions. Due to great advances in their cryptanalysis over the past decade. It is necessary to move away from the underlying constructions most hashing algorithms relied on up until recently: the Merkle Damg\r{a}rd construction.\\
In recent years, significant advances in the field of hash function research have led to the birth of a whole new class of hash functions, based on the sponge construction.\\
With the advance of differencial cryptanalysis,  hashing algorithms such as SHA1 and MD5 are no longer deemed cryptographically secure.\\
And despite the fact the SHA-2 family of functions is still considered secure, inherent weaknesses in the Merkle Damgard underlying constructruction imply that sponge functions such as SHA-3 seem safer than previous ones.\\
Thus it would seem coherent to move directly towards SHA-3, instead of SHA-2. But such a migration is not easy due to legacy systems. Therefore, one wonders how and how fast new cryptographic standards can be applied on existing structures, such as TSL or the Bitcoin cryptocurrency (based on SHA-256). Can such a migration occur without impacting these systems ?\\
\newline

\nocite{*}
\bibliographystyle{alpha}

\begin{frame}[allowframebreaks]
  \frametitle{Livres et références}
  \bibliography{biblio}
\end{frame}

%%%%%%%%%%%%%%%%%%%%%%%%%%%%%%%%%%%%%%%%%%%%%%%%%%%%%%%%%%%%%%%%%%%%%%
\begin{frame}
  \vfill
  \centering
  \highlight{\Huge Questions~?}
  \vfill
\end{frame}
